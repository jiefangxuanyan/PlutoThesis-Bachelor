% !Mode:: "TeX:UTF-8"

\BiChapter{集成学习方法}{}
\label{c:ensemble}
集成学习是一种将多个个体学习器进行组合,而获得更高性能的学习器的机器学习方法。这种方法通常可以得到好于任何一个个体学习器的性能。

主流的集成学习方法分为三种,分别被称为boosting、bagging和stacking。其中,boosting方法使用迭代方法训练一系列学习器,每次迭代都根据上一个学习器的输出改变样本的权重,使得上一个学习器分类错误的样本获得更高权重;Bagging使用一种有放回的“自助采样”方法生成多个样本集,用每个样本集训练一个基学习器,并使用投票方式将这些学习器的输出进行组合;stacking则使用一个次级学习器来组合基学习器的输出。

本课题使用集成学习的方法来结合第\ref{c:classifer}至\ref{c:dict lib}章给出的三种方法的输出。由于各个基学习器的形式不同,本课题使用了stacking方法。