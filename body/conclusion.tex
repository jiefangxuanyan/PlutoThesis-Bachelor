% !Mode:: "TeX:UTF-8" 

\BiAppendixChapter{结\quad 论}{Conclusions}

本课题得到了一套系统完整的中文词语相似度计算方法,其性能在已知的方法中处于较高的地位。高性能的词语相似度计算将会为多种下游自然语言处理问题提供可靠的保障。

深度学习的二分类方法创新性地将无监督回归问题转化为了有监督分类问题,使得广泛存在于互联网中的生语料库可以被用于词语相似度计算问题中。深度学习辅助词嵌入方法利用词嵌入技术解决了词语相似度计算问题,并使用双向LSTM改进了已有的ivLBL方法,充分将LSTM的序列学习能力运用到上下文向量的生成中,扩大了上下文的学习范围,使得词嵌入的质量得到显著的提升。基于或知识库的方法与集成学习方法相结合,改变了传统的基于规则组合特征的方式。集成学习方法有效地组合了上述方法的输出,并得到了比每个个体学习器更强的性能。

然而,本方法也有一定的不足之处。其中一个问题在于本方法无法处理词典外词汇,对于他们只能使用平均值等数据来填充。另一个问题在于集成学习的性能严重受制于有标注数据集的大小和质量。随着日后中文词语相似度标注数据集的增多,本方法的性能也会得到更大的提升。