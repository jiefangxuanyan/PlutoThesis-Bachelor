% !Mode:: "TeX:UTF-8"

\BiChapter{深度学习的二分类方法}{}
\BiSection{原理介绍}{}
为了更清晰的介绍深度学习的二分类方法,本文将先给出一些形式化的定义。

在深度学习的二分类方法中使用的语料库$\mathcal{C}$是一个句子的多重集合,其中每一个句子是一个词语的序列,即:
\begin{equation}
\mathcal{C} \subset \mathcal{V}^+
\end{equation}
其中$\mathcal{V}$是所有词语的集合。

替换函数$f_\text{r}$可以将一个句子$s \in \mathcal{C}$中的某个词语$s_i$替换为一个不同的词语$v \ne s_i$,即:
\begin{equation}
f_\text{r}(s, i, v) = (s_1, \dots, s_{i - 1}, v, s_{i + 1}, \dots, s_{|s|})
\end{equation}
对语料库中的句子应用替换函数$f_\text{r}$得到的句子称为替换句:
\begin{equation}
\mathcal{C}_\text{r} = \bigl\{f_\text{r}(s, i, v) \bigm| s \in \mathcal{C}, v \in \mathcal{V}\bigr\}
\end{equation}
一般来说,$\mathcal{C}_\text{r} \cap \mathcal{C} \approx \emptyset$。本课题希望得到一个二分类器$M_\mathcal{C}$,满足:
\begin{equation}
M_\mathcal{C}(s) = 
\begin{cases}
1 & s \in \mathcal{C}, \\
0 & s \in \mathcal{C}_\text{r}
\end{cases}
\end{equation}
使得$M$可以区分原句与替换句。

利用这样的二分类器,我们就可以定义词语对$(v_\text{a}, v_\text{b})$的相似度:
\begin{equation}
\similarity(w_\text{a}, w_\text{b}) = E\Bigl(M_\mathcal{C}\bigl(f_\text{r}(s, i, v_\text{b})\bigr) \Bigm| s \in \mathcal{C}, s_i = v_\text{a}\Bigr)
\end{equation}
其中$E$为数学期望。其中的思想是,利用包含词语$v_\text{a}$的句子以$v_\text{b}$替换得到的替换句测试二分类器$M_\mathcal{C}$,若二分类器容易将这样的替换句误判为原句,则说明词语对$(v_\text{a}, v_\text{b})$相似。为了提高准确性,应同时考虑从$v_\text{a}$到$v_\text{b}$和从$v_\text{b}$到$v_\text{a}$的替换。

\BiSubsection{训练数据生成}{}
为了训练二分类器$M_\mathcal{C}$,我们需要一些带有标注的句子$(x, y) \in \mathcal{V}^+ \times {0, 1}$作为训练数据。其中:
\begin{equation}
	y = 
	\begin{cases}
		1 & x \in \mathcal{C}, \\
		0 & x \in \mathcal{C}_\text{r}
	\end{cases}
\end{equation}
其中,原句是非常容易生成的,只需将$M_\mathcal{C}$中全部的句子标注为$1$即可。而替换句的生成需要一些额外的步骤。

利用随机采样的方法,我们可以生成一些采样句。采样句定义如下:
\begin{equation}
\mathcal{C}_\text{s} = \Bigl\{f_\text{r}(s, i, v) \Bigm| s \in \mathcal{C}, i \sim \mathcal{U}\bigl\{1, |s|\bigr\}, v \sim \mathcal{D}_\mathcal{V}\Bigr\}
\end{equation}
其中$\mathcal{U}$为均匀分布,$\mathcal{D}_\mathcal{V}$为词语在自然语言中出现的概率分布。也就是说,采样句中的被替换词语位置$i$是随机选择的,而替换词语$v$亦是从词语分布中采样得到的。

显然,$\mathcal{C}_\text{s} \subset \mathcal{C}_\text{r}$,而随机采样对于计算机来说是轻而易举的任务。这样,我们就可以通过生成采样句的方式获得用于训练的替换句。

\BiSubsection{深度学习模型}{}
上文提到,本课题需要训练一个二分类器$M_\mathcal{C}$用于区分原句和替换句。本课题使用了深度学习的模型来完成此任务。