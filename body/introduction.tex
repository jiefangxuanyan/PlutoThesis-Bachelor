% !Mode:: "TeX:UTF-8" 

\BiChapter{绪论}{}
\BiSection{问题背景}{}
词语的相似度计算是自然语言处理中的一项基本任务。它的目的是使用自动化的方法给出一个量化的指标,来衡量一对词语之间的相似程度。词语相似度计算对于很多高层次的自然语言处理任务有着重要的意义,例如问答、信息检索、改写检测与文本蕴含。

为了评价一种词语相似度计算方法的好坏,可以使用内部任务评价或外部任务评价。内部任务评价直接使用一个标准的词语相似度标注数据集,并使用一些评价指标来衡量机器给出的结果与标准数据集之间的差异;而外部任务评价则利用词语相似度计算的结果解决其他的高层次机器学习问题,通过衡量不同计算结果在这些问题上的性能来简介衡量词语相似度计算方法的好坏。

什么样的词语可以被称为“相似”的,这本身并不是一个定义良好的问题。词语的相似度非常依赖于人的主观判断。(写写每个数据集都是怎么搞的。)

本课题取自NLPCC-ICCPOL 2016会议中的“中文词语相似度计算”开放任务。该开放任务使用内部任务评价的方式来衡量各参赛队提交的结果。其组织者发布了一个PKU-500数据集,其中包含500个词语对,以及对每个词语对之间的相似度标注。共有49个参赛队伍报名参加了此次开放任务,而最终由21个参赛队伍提交了总计24个系统。

\BiSection{相关工作}{}
在英文词语相似度任务中,已经有多个数据集可以作为基准测试。其中,使用最广泛的是WordSim-353数据集。这个数据集包含353个词语对,并由16名标注者对其相似度进行了标注。

在NLPCC-ICCPOL 2016会议举办之前,Semeval-2012会议上也举行了一次中文词语相似度的比赛,比赛所使用的词语对是经过翻译的WordSim-353数据集,并重新进行了人工标注。遗憾的是,这次比赛只有两个队伍参加。

本次开放任务中的参赛队伍使用的方法大致分为三类:基于辞典或知识库的方法、基于语料的方法与基于信息检索的方法。

基于词典或知识库的方法利用了一些现有的语言学资源,例如知网(HowNet)和同义词词林。这些词典或知识库的作者往往同时给出了用于计算词语相似度的规则。有些参赛队伍直接使用了这些规则,还有一些对这些规则进行了修改。这些方法依赖于词典或知识库的组织结构,并且无法处理未登录词。此外,人工制定的规则是否合理也是影响模型性能的重要因素。

基于语料的方法主要是根据上下文信息进行词嵌入,这类方法假设“相似的词语应该出现在相似的上下文中”。典型的词嵌入方法,例如word2vec,具有成熟的工具包,因此在各参赛队伍中非常受到欢迎。除了词嵌入方法以外,来自大连理工大学的参赛队使用了深度学习的方法来识别描述词语类比的句子,例如“寂寞和孤单的区别是什么”。如果这类句子出现在语料中,很可能意味着这两个词语是相关的。基于语料的方法需要大规模的语料库和漫长的训练时间,由于这些模型普遍使用无需人工标注的生语料库,因此语料的获取来源非常广泛。而训练时间的缩短只能通过使用运算能力更强的硬件,因此实验可能受到硬件条件的制约。针对中文语料而言,绝大部分的模型是以词为单位进行处理,这样分词的准确度也会对模型最终性能产生一定影响。

基于信息检索的方法利用搜索引擎查询包含目标词语的页面,根据同时包含两个词语的页面数、仅包含一个词语的页面数利用公式计算目标词语的相似度。这种方法的实现非常的简单,只需爬取搜索引擎的结果页面,或调用搜索引擎提供的API。有的队伍将这种方法作为多种评价方法之一,也有的队伍用信息检索得到的结果作为弱监督学习的指标。这种方法的一大问题是词语相似度和词语共同出现的频率并不一致,例如一些事物的别名经常在科普类的文章中被介绍(例如“西红柿”和“番茄”),而因口语习惯而导致不同的同义词(例如“拖后腿”和“拉后腿”)就很难在一篇文章中同时出现。

\BiSection{数据生成与评价指标}{}
PKU-500数据集的生成过程分为三个步骤,分别为词语选择、词语对生成和相似度标注。

PKU-500数据集的词语选择过程遵循以下的条件:
\begin{itemize}
	\item[领域] 同时涉及传统书面语言与近期的网络语言。
	\item[频率] 应同时包含高频、中频与低频的词语。
	\item[词性] 应同时包含名词、动词与形容词。除上述实词外还应包含虚词(例如副词和连词)。
	\item[字数] 应包含1--4个字的词语。
	\item[词义] 应包含部分多义词。
	\item[极性] 应同时包含正面词语与负面词语。
\end{itemize}
根据上述条件,词语的选择过程如下:
\begin{enumerate}
	\item 数据来源于两个领域:三个月的《人民日报》新闻与一大批新浪微博数据。
	\item 数据经过开源的ansj工具进行分词和词性标注。
	\item 从两个领域中分别根据其频率、词性与字数抽取词语。
	\item 根据词义与极性手动挑选自动抽取的词语,最终从《人民日报》新闻与微博数据中分别得到了514和202个词语。
\end{enumerate}

\BiSection{本课题研究方法}{}
本课题的主要研究内容是深度学习方法和集成学习方法。上述两种方法将直接用于词语相似度计算,或用于改进与结合现有的方法。(这里要改)

\BiSubsection{深度学习方法}{}
本课题中使用的深度学习方法是一种基于语料的方法。本课题利用循环神经网络及其变种构造分类器。语料库中的所有句子经过一定的处理之后作为输入,其中将句子不加变化地输入循环神经网络作为正例,而将每个句子中挑选一个词语替换成其它的词语作为负例。分类器的训练目标是尽可能准确地识别被修改的句子。最终评价词语相似度时,将目标词语所在的句子中的目标词语替换成词语对中的另一个词语,并利用上文中的分类器进行识别,使用分类器输出的结果和置信度作为词语相似度的评价标准。

此外,课题还将使用深度学习方法辅助词嵌入的训练。这种方法训练一个带有嵌入矩阵的双向循环神经网络,用来解决词预测问题。令神经网络通过词语两侧的上下文信息补全每一个空缺词。为了解决需要在整个单词表的范围内做规范化的问题,可以使用噪声对比估计的方式。

\BiSubsection{集成学习方法}{}
词语相似度的计算具有多种方法,为了结合这些方法的优势,提高模型的准确率与稳定性,需要利用集成学习的方法。集成学习的方法是通过一些结合策略,将多个个体学习器的输出结合成最终的输出结果。
集成学习中最常见的方法是简单平均法和加权平均法。除此之外,在其他参赛选手的系统中还出现了很多人为制定的基于规则的结合策略。这些方法虽然简单朴素,甚至有些看起来缺乏理论根据,但是在词语相似度计算的任务中取得了相当优秀的成绩。因此作为对比,本课题也会对这些方法进行尝试。
另外一种比较高级的集成学习方法是学习法,这种方法使用另一个学习器来进行结合。这个学习器以每个初级学习期的输出作为输入,而输出一个单个的结果。这样就可以使用初级学习器在测试集上的输出和测试集的标注训练次级学习器。由于官方给出的测试集规模较小,因此不能使用基于深度学习的方法构造次级学习器。本课题将会选用较为简单的模型用于集成学习。
由于可供测试的数据只有宝贵的500个词语对,本课题将使用交叉验证甚至留一法的方式训练次级学习器。这样所有的数据既能参与训练,又能用于评测。而且,进行交叉验证所训练出的学习器还可以再次使用简单平均法集成,形成一个最终的模型。

\BiSubsection{基于词典或知识库的方法的改进}{}
基于词典或知识库的方法往往利用词典和知识库作者提供的相似度计算规则。本课题将会利用词典或知识库的结构提取出特征,然后用机器学习的方式计算相似度。由于这里也需要使用测试集来训练,因此同样需要选择更为简单的模型。
由于基于词典或知识库的学习器和集成学习的次级学习器相近,因此也可以用一个统一的学习器,输入深度学习模型的输出与词典或知识库中的特征,输出最终的结果。